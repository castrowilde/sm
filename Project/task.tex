\documentclass[fleqn,12pt]{article}
\usepackage{amssymb,latexsym}
\usepackage{amsmath,cmap} % cmap позволява търсене на думи на кирилица
\usepackage[T2A]{fontenc} % кодировка за шрифтове на кирилица
\usepackage[cp1251]{inputenc} % кодировка windows-1251 за кирилица
\usepackage[english,bulgarian]{babel} % писане на български и английски език, последният език е текущ
\usepackage[margin=0.8in]{geometry}
\usepackage{titling}
\usepackage{mathtools}
\mathtoolsset{showonlyrefs}
\date{25 април 2017}
\begin{document}
\textit{Задача:} \\
Да се изследва сходимостта на приближеното решение на МКМ на
\[
\int \int\limits_V \int z \sqrt{x^2 + y^2 + z^2}\ dxdydz,
\]
където $V$ се дефинира от неравенствата:
\[
\sqrt{x^2 + y^2} \leq z \leq \sqrt{1 - x^2 - y^2}
\]
\textit{Упътване:} Да се използват сферични координати.\\
\par
\textit{Решение:} \\
\\
\begin{tabular}{|l}
$x = \rho \sin \psi \cos \varphi$ \\
$y = \rho \sin \psi \sin \varphi$ \\
$z = \rho \cos \psi$ \\
$0 \leq \psi \leq \pi, 0 \leq \varphi \leq 2\pi$
\end{tabular}

\begin{multline}
%\end{alignat*}
%\begin{alignat*}{1}
\text{Така, за първото ограничение имаме:}\\
\sqrt{\rho^2 \sin^2 \psi \cos^2 \varphi + \rho^2 \sin^2 \psi \sin^2 \varphi} \leq \rho \cos \psi\\
\sqrt{\rho^2 \sin^2 \psi (\cos^2 \varphi + \sin^2 \varphi)} \leq \rho \cos \psi\\
\sqrt{\rho^2 \sin^2 \psi} \leq \rho \cos\psi\\
\rho \sin \psi \leq \rho \cos \varphi\\
\sin \psi \leq \cos \psi\\
\Rightarrow \psi \in [0;\frac{\pi}{4}]\\
\\
%\end{align}
%\begin{align}
\text{За второто ограничение:}\\
\rho \cos \psi \leq \sqrt{1 - \rho^2 \sin^2 \psi \cos^2 \varphi - \rho^2 \sin^2 \psi \sin^2 \varphi}\\
\rho \cos \psi \leq \sqrt{1 - \rho^2 \sin^2 \psi(\cos^2 \varphi + \sin^2 \varphi)}\\
\rho \cos \psi \leq \sqrt{1 - \rho^2 \sin^2 \psi}\\
\rho^2 \cos^2\psi \leq 1 - \rho^2 \sin^2 \psi\\
1 - \rho^2 \sin^2 \psi - \rho^2 \cos^2 \psi \geq 0\\
1 - \rho^2 (\sin^2 \psi + \cos^2 \psi) \geq 0\\
1 - \rho^2 \geq 0 \Rightarrow \rho^2 \leq 1 \\
\Rightarrow \rho \in [-1;1], \text{но } \rho \geq 0 \Rightarrow \rho \in [0;1]\\
\\
\end{multline}

За Якобиана $J$ имаме $|J| = \rho^2 \sin \psi$.\\
Пресмятаме интеграла:
\begin{multline}
\int\limits_{0}^{1} \int\limits_{0}^{\frac{\pi}{4}} \int\limits_{0}^{2\pi} \Big(\rho^2 \sin \psi . \rho \cos \psi
\sqrt{\rho^2 \sin^2 \psi \cos^2 \varphi + \rho^2 \sin^2 \psi \sin^2 \varphi + \rho^2 \cos^2 \psi}\Big)\, d\varphi d\psi d\rho =\\
= \int\limits_{0}^{1} \int\limits_{0}^{\frac{\pi}{4}} \int\limits_{0}^{2\pi} \Big(\rho^2 \sin \psi . \rho \cos \psi
\sqrt{\rho^2 \sin^2 \psi (\cos^2 \varphi + \sin^2 \varphi) + \rho^2 \cos^2 \psi}\Big)\, d\varphi d\psi d\rho =\\
= \int\limits_{0}^{1} \int\limits_{0}^{\frac{\pi}{4}} \int\limits_{0}^{2\pi} \Big(\rho^3 \sin \psi \cos \psi
\sqrt{\rho^2 (\sin^2 \psi + \cos^2 \psi)}\Big)\, d\varphi d\psi d\rho =\\
= \int\limits_{0}^{1} \int\limits_{0}^{\frac{\pi}{4}} \int\limits_{0}^{2\pi} \Big(\rho^4 \sin \psi \cos \psi \Big) d\varphi d\psi d\rho =
 2\pi \int\limits_{0}^{1} \rho^4 \, d\psi \, \int\limits_{0}^{\frac{\pi}{4}}(\sin \psi \cos \psi)\, d\psi =\\
= \left. 2\pi \cdot \frac{\rho^5}{5} \right\vert_{0}^{1} \cdot \frac{1}{4} =
2\pi \cdot \frac{1}{5} \cdot \frac{1}{4} = \frac{2\pi}{20} = \frac{\pi}{10} \thickapprox 0,31415
\end{multline}
\end{document} 
